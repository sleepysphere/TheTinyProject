\chapter{Conclusion}
\label{chap:conclusion}

This project met the objectives set out in the specification, covering both the creation of a basic C++ linear algebra library and its use in a linear regression scenario.

\section{Summary of Accomplishments}
\begin{itemize}
    \item \textbf{Part A: Core Linear Algebra Classes:}
    \begin{itemize}
        \item Developed a robust \texttt{Vector} class with dynamic memory handling, overloaded operators for standard arithmetic (including scalar operations), and both 0-based and 1-based indexing with bounds checking.
        \item Built a comprehensive \texttt{Matrix} class supporting memory management, multiple constructors (including deep copy), operator overloading for various matrix operations, and methods for determinant, inverse (for square matrices), and Moore-Penrose pseudo-inverse (for general matrices). This enables handling both regular and irregular systems, including non-square matrices.
        \item Implemented a \texttt{LinearSystem} class to solve $Ax=b$ for square matrices using Gaussian elimination with partial pivoting, with careful attention to class design and memory management.
        \item Created a derived \texttt{PosSymLinSystem} class for efficiently solving symmetric positive definite systems via the Conjugate Gradient method, demonstrating polymorphism by overriding the virtual \texttt{Solve} method and including symmetry checks.
    \end{itemize}
    \item \textbf{Part B: Linear Regression Application:}
    \begin{itemize}
        \item Applied the developed classes to perform linear regression on the UCI Computer Hardware dataset.
        \item Selected relevant features, split the data into training (80\%) and testing (20\%) sets, and formulated the regression problem using normal equations.
        \item Used the \texttt{LinearSystem} class to solve for model parameters.
        \item Evaluated model performance using Root Mean Square Error (RMSE) on both training and testing sets, highlighting differences that may indicate overfitting or the limitations of a simple linear model for this dataset.
    \end{itemize}
\end{itemize}

\section{Challenges and Lessons Learned}
This project offered valuable experience in C++ programming and numerical methods:
\begin{sloppypar}
\begin{itemize}
    \item \textbf{Object-Oriented Design:} Gained experience designing interconnected classes (\texttt{Vector}, \texttt{Matrix}, \texttt{LinearSystem}, \texttt{PosSymLinSystem}) with proper encapsulation, inheritance, and polymorphism.
    \item \textbf{Memory Management:} Practiced dynamic memory allocation (\texttt{new}, \texttt{delete}, \texttt{delete[]}) and learned the importance of constructors, destructors, and copy semantics (Rule of Three/Five).
    \item \textbf{Operator Overloading:} Developed intuitive interfaces for mathematical objects through operator overloading.
    \item \textbf{Numerical Algorithms:} Implemented key numerical linear algebra algorithms such as Gaussian elimination with pivoting, the Conjugate Gradient method, and methods for determinant, inverse, and pseudo-inverse, while considering their mathematical foundations and numerical stability.
    \item \textbf{Machine Learning Application:} Connected a custom numerical library to a practical data analysis task, reinforcing the role of linear algebra in machine learning techniques like linear regression.
\end{itemize}
\end{sloppypar}
A significant challenge was ensuring numerical stability and correctness, especially for matrix inversion and solving systems. Debugging memory management and operator behavior also required careful attention.

\section{Opportunities for Future Work}
This project provides a solid foundation for further enhancements. Possible future directions include:
\begin{itemize}
    \item \textbf{Advanced Numerical Methods:} Adding more sophisticated matrix decompositions for improved stability or efficiency.
    \item \textbf{Sparse Matrix Support:} Extending the library to efficiently handle sparse matrices with specialized storage and algorithms.
    \item \textbf{Templated Classes:} Making \texttt{Vector} and \texttt{Matrix} class templates to support various numerical types.
    \item \textbf{Improved Error Handling:} Replacing assertions with a more robust exception handling system.
    \item \textbf{Regularization Methods:} Implementing techniques like Tikhonov regularization or Ridge regression within the solvers or regression framework.
    \item \textbf{Expanded Machine Learning Applications:} Enhancing the application with more thorough model evaluation, feature scaling, or comparisons with other regression models.
\end{itemize}

In summary, this project demonstrates the ability to design, implement, and apply a C++ library for essential linear algebra operations. It fulfills the requirements and offers meaningful experience in both software development and numerical computation.